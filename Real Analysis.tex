\documentclass{report}
\usepackage{amsmath}
\usepackage{amsfonts}
\usepackage{physics}
\begin{document}
\title{Real Analysis}

\newtheorem{definition}{Definition}[section]
\newtheorem{axiom}{Axiom}[section]
\newtheorem{theorem}{Theorem}[section]

\maketitle
\chapter{Relations}
\section{Relations}

\begin{definition}
Let X and Y be sets. Then $R\subseteq$ $X\times$Y is called a \textbf{Relation} over sets X and Y.
\end{definition}
If $(x,y)\in $R, its is denoted as xRy, or "x is related to y", where $x \in $X and $y \in $Y. Sometimes, relations over two sets are called binary relations as in general, relations can be subsets of Cartesian Product of more than two sets.

\section{Functions}
\begin{definition}
A Relation is called \textbf{functional} or right unique if:\\
$\forall $x $\in $X , $\forall $y $\in $Y,$\forall $z $\in $Y, $(x,y)\in $R $\wedge$ $(x,z)\in $R $\implies$ $y=$z
\end{definition}

\begin{definition}
A Relation is called \textbf{serial} or left total if:\\
$\forall $x$\in $X, $\exists $y$\in $Y:$(x,y) \in $R
\end{definition}

\begin{definition}
A Relation that is functional and serial is called a \textbf{function}.
\end{definition}
A function is usually denoted by\\
\\
\emph{f} : X $\rightarrow$ Y\\
\\
and is usually thought of as a rule that assigns every element in X to exactly one element in Y.

\begin{definition}
A function that is left unique is called \textbf{injective} or one-one.
\end{definition}

\begin{definition}
A function that is right total is called \textbf{surjective} or onto.
\end{definition}

\begin{definition}
A function that is both one-one and onto is called a \textbf{bijective function} or a \textbf{bijection}.
\end{definition}
If a function \emph{f} is bijective, we its \textbf{inverse} as\\
\\
$f^{-1}$: Y $\rightarrow$ X\\

\section{Homogeneous Relations}
\begin{definition}
A \textbf{Homogeneous Relation} is a subset of a Cartesian Product of a set with itself.
\end{definition}

\begin{definition}
A Relation R is \textbf{Reflexive} if \\
$\forall$x $\in$X, (x,x) $\in$ R
\end{definition}

\begin{definition}
A Relation is \textbf{Symmetric} if \\
$\forall$x,y $\in$X, (x,y) $\in$ R $\implies$ (y,x) $\in$ R
\end{definition}

\begin{definition}
A Relation is \textbf{Transitive} if \\
$\forall$x,y,z $\in$X, (x,y) $\in$ R $\wedge$ (y,z) $\in$ R $\implies$ (x,z) $\in$ R
\end{definition}

\begin{definition}
A Relation is \textbf{Antisymmetric} if \\
$\forall$x,y $\in$X, (x,y) $\in$ R $\implies$ (y,x) $\notin$ R
\end{definition}


\section{Equivalence Relations}
\begin{definition}
A Relation that is Reflexive, Symmetric and Transitive is called an  \textbf{Equivalence Relation}.
\end{definition}
Equivalence relations are usually denoted by $\sim$.
Let $\sim$ be an equivalence relation on a set X. If x,y $\in$ X are related by  $\sim$ then it is denoted by x$\sim$y.
\begin{definition}
Let $\sim$ be an equivalence relation defined on a set X. \\Let x $\in$ X. Then the set \\
$\left[x\right]$ := \{y $\vert$ y $\sim$ x\} \\ is called the \textbf{equivalence class} containing x.
\end{definition}

Equivalence classes have the following properties - 
\begin{enumerate}
\item Let x,y $\in$ X, then either $\left[x\right]$ = $\left[y\right]$ or $\left[x\right]$ $\cap$ $\left[y\right]$ = $\emptyset$
\item $\bigcup_{x \in X}$ = X.
\end{enumerate}
Thus equivalence classes form a partition of the set X.

\section{Partial Orders}
\begin{definition}
A Relation that is Reflexive, Antisymmetric and Transitive is called a  \textbf{Partial Order}.
\end{definition}

\begin{definition}
A set A $\subseteq$ X is said to be \textbf{bounded above} if \\$\exists$b $\in$ X : $\forall$a $\in$ A  b $\geq$ a.\\
In this case b is said to be an \textbf{upper bound} of A.
\end{definition}

\begin{definition}
Let X be a set and A $\subseteq$ X. The s $\in$ X is called a \textbf{least upper bound} of A if it satisfies the following conditions
\begin{enumerate}
\item s an upper bound of A
\item If b is an upper bound of A, then s $\geq$ b.
\end{enumerate}
It is also known as the \textbf{supremum} of A. It is usually denoted by s = lub A or s = sup A.
\end{definition}

\chapter{Equinumerosity and Countability}

\chapter{Real Numbers}
\begin{axiom}[Axiom of Completeness]
Every nonempty subset of $\mathbb{R}$ that is bounded above has a least upper bound.
\end{axiom}
The Axiom of Completeness is one of the defining characteristics of real numbers. Many important theorems about real numbers follow as a consequence of the Axiom of Completeness.

\begin{theorem}[Nested Interval Property]
For each n $\in \mathbb{N}$, consider the closed intervals $I_n$ = $\left[a_n,b_n\right] = \{ x \in \mathbb{R}  \vert  a_n \leq x \leq b_n\}$ such that $I_1 \subseteq I_2 \subseteq \cdots$. Then $\bigcap^\infty_{n=0}I_n \neq \emptyset$
\end{theorem}
The Nested interval property is just another way of expressing the completeness of $\mathbb{R}$ which will be useful to prove some theorems in real analysis. There are other ways of expressing the same which we will explore in the coming chapters.

\chapter{Sequences and Series of Real Numbers}
\section{Sequences}

\begin{definition}
A \textbf{sequence} is a function whose domain is $\mathbb{N}$.
\end{definition}

\begin{definition}
A sequence $a_n $ is said to \textbf{converge} to a limit a if \\
$\forall$ $\epsilon$ $>$ 0 , $\exists$N $\in$ $\mathbb{N}$:$\forall$ n$\geq$ N , $\lvert a_n - a \rvert$ $<$ $\epsilon$
\end{definition}

\begin{definition}
A sequence $x_n$ is said to be \textbf{bounded} if \\
 $\exists$ M $>$ 0 : $\forall$ n $\in$ $\mathbb{N}$ , $\lvert x_n \rvert$ $<$ M
\end{definition}

\begin{theorem}
All convergent sequences are bounded.
\end{theorem}

\begin{theorem}[Algebraic Limit Theorem]
Let $a_n$ and $b_n$ be sequences, such that $a_n$ $\rightarrow$ a and $b_n$ $\rightarrow$ b. Then,
\begin{enumerate}
\item $a_n$ + $b_n$ $\rightarrow$ a + b
\item $ca_n$ $\rightarrow$ ca
\item $a_n b_n$ $\rightarrow$ ab
\item $\frac{a_n}{b_n}$ $\rightarrow$ $\frac{a}{b}$, b $\neq$ 0
\end{enumerate}
\end{theorem}

\begin{theorem}
Let $a_n$ and $b_n$ be sequences, such that $a_n$ $\rightarrow$ a and $b_n$ $\rightarrow$ b. Then,
\begin{enumerate}
\item If $a_n$ $\geq$ 0 $\forall$ n, then a $\geq$ 0
\item If $a_n$ $\geq$ $b_n$ $\forall$ n, then a $\geq$ b
\item If $\exists$ c $\in$ $\mathbb{R}$, such that c $\leq$ $b_n$ $\forall$ n $\in$ $\mathbb{N}$, then c $\leq$ b. Similarly, if c $\geq$ $a_n$ \\ $\forall$n $\in$ $\mathbb{N}$, then c $\geq$ a
\end{enumerate}
\end{theorem}
\begin{definition}
A sequence is increasing if $a_{n+1}$ $\geq$ $a_n$ $\forall$ n $\in$ $\mathbb{N}$ and decreasing if  $a_{n+1}$ $\leq$ $a_n$ $\forall$ n $\in$ $\mathbb{N}$. \\
A sequence is \textbf{monotone} if it is increasing or decreasing.
\end{definition}

\begin{theorem}[Monotone Convergence Theorem]
If a sequence is bounded and monotone, it converges.
\end{theorem}

\begin{theorem}[Bolzano Weierstrass Theorem]
Every bounded sequence has a convergent subsequence.
\end{theorem}

\begin{definition}
A sequence is \textbf{Cauchy} if\\
$\forall$ $\epsilon$ $>$ 0 , $\exists$ N $\in$ $\mathbb{N}$ : $\forall$ m,n $\geq$ N, $\lvert a_m - a_n \rvert$ $<$ $\epsilon$.
\end{definition}

\begin{theorem}
All convergent sequences are Cauchy sequences.
\end{theorem}

\begin{theorem}
All Cauchy sequences are bounded.
\end{theorem}

\begin{theorem}[Cauchy Criterion]
A sequence in convergent if and only if it is Cauchy.
\end{theorem}
The Axiom of Completeness, Nested Interval Property, Monotone Convergence Theorem, Bolzano Weierstrass Theorem and Cauchy's Criterion all assert the same fact - The completeness of $\mathbb{R}$ - in their own language. Any one of them can be taken as an axiom and used to prove the rest.

\section{Series}
\begin{definition}
Let $x_n$ be a sequence. A series is an expression of the form \\ 
$\sum_{n=1}^{\infty}$ $x_n$ = $x_1$ + $x_2$ + $\cdots$ is called a \textbf{series}.
\end{definition}

\begin{definition}
The sequence $s_n$ = $x_1$ + $x_2$ + $\cdots$ + $x_n$ is called the \textbf{sequence of partial sums}.
\end{definition}
The series $\sum_{n=1}^{\infty}s_n$ is said to converge to a s if the corresponding sequence of partial sums $s_n$ converges to s.\\
$\sum_{n=1}^{\infty}$ = s $\leftrightarrow$  $s_n$ $\rightarrow$ s.

\begin{theorem}[Algebraic Limit Theorem for Series]
Let $\sum_{n=1}^{\infty}a_n = A,\sum_{n=1}^{\infty}b_n = B$. Then 
\begin{enumerate}
\item $\sum_{n=1}^{\infty}ca_n = cA$ $\forall c \in \mathbb{R}$.
\item $\sum_{n=1}^{\infty}(a_n + b_n) = A + B$.
\end{enumerate} 
\end{theorem}

\begin{theorem}[Cauchy Criterion for Series]
The series $\sum_{k=1}^{\infty}a_k$ converges if and only if  \\ $\forall \epsilon > 0, \exists N \in \mathbb{N}$ such that when $n \geq m \geq N$, $\lvert a_{m+1} + a_{m+2} + \cdots + a_n\rvert < \epsilon$.
\end{theorem}

\begin{theorem}
If the series  $\sum_{k=1}^{\infty}a_k$ converges, then $a_k \rightarrow 0$.
\end{theorem}

\begin{theorem}[Comparison Test]
Let $a_k,b_k$ be sequences satisfying $0 \leq a_k \leq b_k$ $\forall k \in \mathbb{N}$, then 
\begin{enumerate}
\item If  $\sum_{k=1}^{\infty}b_k$ converges, then  $\sum_{k=1}^{\infty}a_k$ converges.
\item  $\sum_{k=1}^{\infty}a_k$ diverges, then  $\sum_{k=1}^{\infty}b_k$ diverges.
\end{enumerate}
\end{theorem}
\begin{theorem}[Absolute Convergence Test]
If the series $\sum_{n=1}^{\infty}\lvert a_n \rvert$ converges, then the series $\sum_{n=1}^{\infty}a_n$ also converges.
\end{theorem}

\begin{definition}
If the series $\sum_{n=1}^{\infty}\lvert a_n \rvert$ converges, then the series $\sum_{n=1}^{\infty}a_n$ is said to converge \textbf{absolutely}. If  $\sum_{n=1}^{\infty}a_n$ converges but $\sum_{n=1}^{\infty}\lvert a_n \rvert$ diverges, then $\sum_{n=1}^{\infty}a_n$ is said to converge \textbf{conditionally}.
\end{definition}
\end{document}

